\recipe{{Grießbrei mit Obst}}{GriessbreiObst}{
\preptime{ca. 20 \minutes}
\persons{1--2}
\ingredient{1\liter Milch}
\ingredient{4 Esslöffel Zucker}
\ingredient{1 Prise Salz}
\ingredient{Zitronenschale}
\ingredient{125--130\gramm Grieß (Hartweizen oder Weichweizen)}
\ingredient{2 haselnussgroße Stückchen \\ Butter}
\ingredient{Evtl. 1 Ei}
\ingredient{Obst}
}{
\item Milch mit Zucker zum Kochen bringen, Salz und Zitronenschale hinzugeben.
\item Von der Herdplatte nehmen und den Grieß reinrieseln lassen. Mit einem Schneebesen umrühren und auf der ausgeschalteten Herdplatte ca. 10--15 \minutes quellen lassen.
\item Die Butter dazu geben und evtl. ein Ei ganz oder nur das Eigelb einrühren.
\item Frisches Obst (Erdbeeren), frischen Kompott oder Obst aus der Konserve (z.B. vom Vater) dazu reichen.
\item \textbf{Alternativ mit Milchreis:} 1\liter Milch mit Zucker zum Kochen bringen. 200\gramm Milchreis in die kochende Milch geben. Zitronenschale und Prise Salz dazu geben, umrühren.
Die Masse bei ganz geringer Temperatur 30--45 \minutes ausquellen lassen. Aufpassen, dass der Reis nicht anliegt bzw. überkocht. Zwischendurch umrühren. Zubereitungszeit: ca. 50 \minutes. 
}
