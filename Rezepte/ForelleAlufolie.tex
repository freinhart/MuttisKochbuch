\recipe{{Forelle in Alufolie}}{ForelleAlufolie}{
\preptime{ca. 45 \minutes}
\persons{2}
\ingredient{2 mittelgroße, ausgenommene \\ Forellen}
\ingredient{1 Strauß glatte Petersilie}
\ingredient{1--2 Knoblauchzehen}
\ingredient{2 Zitronenviertel}
\ingredient{Salz, Pfeffer, Zitronensaft, Öl}
\ingredient{Alufolie}
}{
\item Forellen waschen, mit Zitronensaft den Bauchraum auströpfeln, salzen, pfeffern und ziehen lassen.
\item Petersilie waschen, Knoblauch schälen und in kleine Scheiben schneiden. Petersilie und Knoblauchscheiben in den Bauch füllen.
\item Alufolie in Stücke trennen und mit Öl bepinseln.
\item Die vorbereiteten Forellen in die Mitte legen und die Folie zusammenrollen und -drücken.
\item Im Backofen bei 200 \degree 20--25 \minutes backen.
\item Mit Zitronenvierteln servieren.
\item Als \textbf{Beilagen} reicht man Salz- oder Pellkartoffeln mit gekochtem Gemüse (Möhren, Broccoli o.a.) oder grünen Salat.
\item \textbf{Variation mit Mandelbutter:} In heißer Butter in einer Pfanne Mandelblättchen sanft bräunen und zur Forelle reichen.
}
