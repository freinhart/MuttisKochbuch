\recipe{{Sauerkraut mit Flönz und Kartoffelpüree}}{Sauerkraut2}{
\preptime{ca. 45 \minutes}
\persons{2--4}
\ingredient{1 Dose Mildessa Sauerkraut \\ (oder anderes)}
\ingredient{1 große Zwiebel}
\ingredient{1 Apfel}
\ingredient{Öl oder Schmalz}
\ingredient{Etwas Speckwürfel}
\ingredient{1--2 Lorbeerblätter}
\ingredient{ca. 4 Wacholderbeeren}
\ingredient{1 Teelöffel Kümmel}
\ingredient{$\frac{1}{4}$\liter Gemüsebrühe (instant)}
\ingredient{1 Esslöffel Honig}
\ingredient{Pfeffer}
\ingredient{$\frac{1}{2}$--1 Ring geräucherte Blutwurst}% \\ im Naturdarm}
\ingredient{Mehl zum Stäuben}
\ingredient{Senf}
\ingredient{Kartoffeln, Salz, Milch, Muskat}
}{
\item Kartoffeln schälen und in Würfel schneiden.
\item Salzkartoffeln kochen.
\item Für das Sauerkraut die Zwiebel und den Apfel klein schneiden. 
\item Öl oder Schmalz erhitzen und Zwiebeln, Speckwürfel und Apfel andünsten. 
\item Das Sauerkraut dazugeben und mit der Brühe ablöschen. 
\item Gewürze dazu geben und 15 \minutes köcheln lassen. Mit Honig abschmecken.
\item Das Püree aus den fertig gekochten Salzkartoffeln zubereiten. Kartoffeln abgießen und mit etwas Milch im Topf erwärmen. 
\item Mit den Schneebesen und dem Mixer alles in einem Rührbecher zusammen mit Muskat und Salz schlagen. Evtl. noch Milch hinzufügen, falls das Püree zu fest ist.
\item In einer Pfanne das Öl erhitzen und die in Scheiben geschnittene und mit Mehl bestäubte Flönz von beiden Seiten kurz und sanft braten.
\item Als \textbf{Variation} auch Nürnberger Rostbratwürstchen oder Leberkäse im Kraut erwärmen oder separat braten.
\item Zu den Fleischbeilagen Senf reichen.
}
