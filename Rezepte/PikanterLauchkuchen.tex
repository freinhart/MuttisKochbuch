\recipe{{Pikanter Lauchkuchen \newline (oder Zucchinikuchen)}}{PikanterLauchkuchen}{
\preptime{Teig ca. 10 \minutes \\\phantom{\Interval} + 1 \hour Ruhezeit}
\preptime{Füllung 30 \minutes}
\preptime{Backzeit 25--30 \minutes}
\vspace{-0.3cm}
\persons{2--3}
\vspace{-0.4cm}
\ingredientsec{Teig}
\ingredient{200\gramm Mehl}
\ingredient{100\gramm kalte Magarine oder Butter}
\ingredient{1 Messerspitze Salz}
\ingredient{1--2 Esslöffel kaltes Wasser}
\ingredient{Öl zum Einfetten}
\ingredientsec{Lauchfüllung}
\ingredient{750\gramm Lauch}
\ingredient{100--125\gramm durchwachsener Speck}
\ingredient{2 Esslöffel Öl}
\ingredient{Etwas Zitronensaft}
\ingredient{Sojasoße, Pfeffer, Salz, Kümmel gemahlen, Muskat, süßes Paprikapulver}
\ingredient{1 Becher saure Sahne oder \\ \cremefraiche}
\ingredient{1--2 Eier}
\ingredient{50\gramm geriebener Käse (oder nach Belieben mehr)}
}{
\item Den Knetteig für eine Springform mit obigen Zutaten vorbereiten und kühl stellen.
\item Den Lauch putzen (waschen, gucken, dass kein Dreck mehr zwischen den einzelnen Schichten ist. 
Dafür u. U. die Lauchstangen der Länge nach halbieren und dann  waschen) und in kleine Stücke schneiden. 
\item Den Teig dünn ausrollen und kleinen Rand für die Springform herstellen. 
Mit der Gabel mehrmals einstechen und im Ofen bei ca. 200 Grad (Umluft 180 Grad) 10 \minutes vorbacken.
\item In der Zeit Speck und Lauch im Öl in einer Pfanne leicht andünsten und mit Zitronensaft, Sojasoße, Kümmel, Salz und Pfeffer abschmecken und etwas abkühlen lassen.
\item In dieser Zeit die Soße mit dem Mixer (Schneebesen) herstellen aus saurer Sahne, Ei und Gewürzen.
\item Auf dem vorgebackenen Boden Lauchfüllung verteilen und mit dem geriebenen Käse bestreuen.
\item Im Ofen bei 200 Grad (Umluft 180 Grad) ca. 25 \minutes backen.
%\setcounter{stepcnt}{-1}
\item Für die \textbf{Zucchinifüllung} die Zucchini waschen Ende abschneiden und in Scheiben schneiden. 
In Öl mit dem Speck anbraten und mit Zitronensaft, Sojasoße, Salz, Pfeffer abschmecken.
Für die Soße statt saurer Sahne süße Sahne verwenden oder auch \cremefraiche, sonst wie oben.
}
