\recipe{{Blumenkohl-Nudel-Auflauf}}{Blumenkohlauflauf}{
\preptime{ca. 1 \hour}
\persons{2--3}
\ingredient{1 Blumenkohl}
\ingredient{250\gramm Nudeln oder Makkaroni}
\ingredient{Salzwasser für die Nudeln}
\ingredient{Salzwasser mit Muskat und etwas Milch für den Blumenkohl}
\ingredient{150\gramm Schinkenspeck}
\ingredient{2 Eier}
\ingredient{$\frac{1}{2}$ Becher \cremefraiche oder \\ süße Sahne}
\ingredient{1 Päckchen helle Soße \\ ($\frac{1}{4}$\liter Flüssigkeit)}
\ingredient{Zitronensaft, Muskat, Paprikapulver}
\ingredient{Geriebener Käse}
\ingredient{Öl oder Butter für die Auflaufform}
}{
\item Den Blumenkohl putzen, vom Strunk abtrennen, in Röschen zerteilen und waschen, abtropfen lassen. 
\item Das Wasser mit Salz und Muskat zum Kochen bringen, Milch hinzufügen (damit der Blumenkohl schön weiß bleibt). 
\item Dann die Blumenkohlröschen ins kochende Wasser legen und bei milder Hitze gar kochen (nicht zu weich!) und im Sieb abtropfen lassen.
\item Gleichzeitig auch das Nudelwasser zum Kochen bringen und die Nudeln nach Anleitung kochen und abschrecken, im Sieb abtropfen lassen.
\item Während der Kochzeit die Soße zubereiten. Dazu die helle Soße nach Anleitung zubereiten und mit Muskat und Zitronensaft abschmecken. 
\item Die Eier und die Sahne/\cremefraiche miteinander mit den Schneebesen verrühren und mit Paprikapulver leicht würzen. 
\item Beide Soßen miteinander vermischen, die helle Soße darf dabei nicht kochen!
\item Die Nudeln in die gefettete Auflaufform geben und mit den Speckwürfeln vermischen. 
\item Die Blumenkohlröschen auf den Nudeln verteilen und mit der Soße überziehen. 
\item Zuletzt den geriebenen Käse darüber geben.
\item Im Backofen bei 160--180 \degree 25--30 \minutes backen.
}
